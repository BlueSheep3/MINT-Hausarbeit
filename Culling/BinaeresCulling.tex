\subsection{Binäres Culling}

Wir können beide Probleme der ersten Implementation
mit einer neuen Methode lösen.
Wir betrachten dabei eine Reihe von Voxels als einen
64-Bit Integer, wobei die einzelnen Bits darstellen,
ob sich dort ein Voxel befindet.
Es werden dabei 64 Bits gebraucht und nicht nur 32,
da man für das Culling auch wissen muss,
welche Voxels sich am Rand eines Chunks befinden,
und ein Chunk 32 Voxels lang ist.
Somit könnten die Voxels in einem Chunk
(und dem Rand) als ein 2 dimensionales
Array von 64-Bit Integers dargestellt werden,
wobei die 2 Dimensionen des Arrays die
$y$- und $z$-Achse darstellen und der Integer eine
Reihe von Voxels in der $x$-Achse darstellt.
Der Vorteil davon ist, dass wir somit in der
$x$-Achse binäre Arithmetik anwenden können für
das Culling, was sehr schnell ist.

% TODO better explanation
Beim Culling brauchen müssen wir die Seiten von
Voxels finden, die nicht von einem anderen Voxel
verdeckt werden.
Mit binärer Arithmetik geht dies jetzt sehr einfach:
% TODO better styling of code
\begin{verbatim}
fn find_faces(bits: u32) -> u32 {
    bits & !(bits >> 1)
}
\end{verbatim}
Diese Funktion gibt eine Bitmaske für alle Voxels,
die von links sichtbar sind.
Definiert man also eine Funktion mit \verb|bits << 1|,
dann bekommt man alle, die von rechts sichtbar sind.

Um dies für jede Achse zu wiederholen,
erstellt man so ein Array für jede Achse.
Diese kann man mit nur einem Zugriff pro Voxel
wie folgt erstellen:

% TODO insert code for creating bit masks

% TODO benchmark binary culling
