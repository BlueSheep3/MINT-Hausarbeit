\newcommand{\minipagespace}[1]{
	\begin{minipage}[c]{#1\textwidth}
		\ 
	\end{minipage}
}

\section{Culling}

Der erste Weg eine Voxel Engine zu optimieren wird
offensichtlich, wenn man sich einen Würfel aus $8$
Voxels betrachtet. Dabei beobachten wir, dass
die Hälfte der Seiten der Voxels nach innen schauen
und somit nicht sichtbar sind. Wenn man die
Seitenlänge dieses Würfels verdoppelt, multipliziert
man die Oberfläche mit $4$ und das Volumen mit $8$.
Somit entstehen immer mehr Seiten, die nach innen
schauen, umso größer das Objekt ist. Also wird
dies bei großen Objekten dazu führen, dass die
meisten Seiten nicht sichtbar sind.

\begin{center}
\begin{figure}[ht]
	\minipagespace{0.04}
	\begin{minipage}[c]{0.4\textwidth}
		\begin{center}
\includegraphics[width=1\textwidth]{assets/Culling/Opaque8Blocks.png}
		\end{center}
	\end{minipage}
	\minipagespace{0.09}
	\begin{minipage}[c]{0.4\textwidth}
		\begin{center}
\includegraphics[width=1\textwidth]{assets/Culling/Transparent8Blocks.png}
		\end{center}
	\end{minipage}\hfill
\end{figure}
\end{center}

Mit \gqq{Culling} beschreibt man die Optimierung,
diese Seiten zu entfernen.

{ \subsection{Erste Implementation}

Die jetzige Methode ein Polygonnetz der Voxels
zu erstellen besteht darin, für jeden Voxel
ein Würfelnetz zu erstellen.
Um dies also zu cullen, überprüft man
noch die Nachbarvoxels, um zu entscheiden,
welche Seiten sichtbar sind, und erstellt nur
die sichtbaren Seiten.

% TODO maybe show some of the code here

\begin{figure}[ht]
	\begin{minipage}[c]{0.49\textwidth}
		\begin{center}
\includegraphics[width=0.8\textwidth]{assets/Culling/ChunkBorders.png}
		\end{center}
	\end{minipage}
	\begin{minipage}[c]{0.49\textwidth}
Da die Spielwelt unendlich groß ist, muss sie
in viele einzelne Polygonnetze aufgeteilt werden.
Somit ist die Welt in sogenannte \gqq{Chunks}
eingeteilt, die $32 \times 32 \times 32$ Voxels
beinhalten. Das Polygonnetz, das die Voxels in einem
Chunk anzeigt, werde ich als
\gqq{Chunknetz} bezeichnen.
	\end{minipage}\hfill
\end{figure}

Da das Erstellen eines Chunknetzes lang dauern kann,
soll das Spiel nicht darauf warten,
da dies sonst sichtbar beim Spielen wäre.
Deswegen werden die Chunknetze in separaten
\href{https://de.wikipedia.org/wiki/Thread_(Informatik)}{Threads}
erstellt.
Zudem können somit mehrere Chunknetze gleichzeitig
erstellt werden.

\vspace{0.3cm}

% TODO transition into the benchmarks

% dont know how to get rid of this
% warning or what it means
% Erste Implementation stats from: bench 01
\benchgraph{1}{0.0}{
	algo                   & blue       & red       \\
	Erste Implementation   & 12.834184  & 0.672563  \\
}

\vspace{0.3cm}

Dadurch entsteht aber ein neues Problem: \\
Wenn mehrere Threads zugriff auf die gleichen Daten
haben, muss dieser Zugriff synchronisiert werden,
da sonst eine
\href{https://de.wikipedia.org/wiki/Wettlaufsituation}{Wettlaufsituation}
(genauer gesagt ein Data Race) entstehen kann.
\footnote{\url{https://doc.rust-lang.org/nomicon/races.html}}
Durch diese Synchronisierung müsste aber jeder Zugriff
zu Chunks auf andere Threads warten, was das gesamte
Spiel langsamer machen würde. Deswegen werden die
Daten der nötigen Chunks zu dem Thread rüberkopiert,
was langsam ist, da es sich hier über sehr große Daten
handelt.

Culling braucht Information aus den benachbarten
Chunks, um zu entscheiden, ob die Ränder des Chunks
sichtbar sind. Somit müssen die 6 Nachbarchunks
auch zu dem Thread rüberkopiert werden, was sehr viele
Daten sind. Man könnte zwar nur die Voxels
rüberkopieren, die am Rand des Chunks sind,
aber wir werden gleich eine Methode sehen,
die dieses Problem und andere auf einmal löst.

\vspace{0.3cm}

Ein weiteres Problem besteht darin, dass dieser
Algorithmus für jeden Voxel noch die 6 Nachbarn
betrachten muss. Somit wird jeder Voxel 7-mal
betrachtet.
 }

{ \subsection{Binäres Culling}

Wir können beide Probleme der ersten Implementation
mit einer neuen Methode lösen.
Wir betrachten dabei eine Reihe von Voxels als einen
64-Bit Integer, wobei die einzelnen Bits darstellen,
ob sich dort ein Voxel befindet.
Es werden dabei 64 Bits gebraucht und nicht nur 32,
da man für das Culling auch wissen muss,
welche Voxels sich am Rand eines Chunks befinden,
und ein Chunk 32 Voxels lang ist.
Somit könnten die Voxels in einem Chunk
(und dem Rand) als ein 2 dimensionales
Array von 64-Bit Integers dargestellt werden,
wobei die 2 Dimensionen des Arrays die
$y$- und $z$-Achse darstellen und der Integer eine
Reihe von Voxels in der $x$-Achse darstellt.
Der Vorteil davon ist, dass wir somit in der
$x$-Achse binäre Arithmetik anwenden können für
das Culling, was sehr schnell ist.

% TODO better explanation
Beim Culling brauchen müssen wir die Seiten von
Voxels finden, die nicht von einem anderen Voxel
verdeckt werden.
Mit binärer Arithmetik geht dies jetzt sehr einfach:
% TODO better styling of code
\begin{verbatim}
fn find_faces(bits: u32) -> u32 {
    bits & !(bits >> 1)
}
\end{verbatim}
Diese Funktion gibt eine Bitmaske für alle Voxels,
die von links sichtbar sind.
Definiert man also eine Funktion mit \verb|bits << 1|,
dann bekommt man alle, die von rechts sichtbar sind.

Um dies für jede Achse zu wiederholen,
erstellt man so ein Array für jede Achse.
Diese kann man mit nur einem Zugriff pro Voxel
wie folgt erstellen:

% TODO insert code for creating bit masks

% TODO benchmark binary culling
 }
