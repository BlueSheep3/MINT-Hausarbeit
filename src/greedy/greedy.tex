\section{Greedy Meshing}

Nun kommen wir zum zweiten Weg eine Voxel Engine
zu optimieren:

\begin{center}
\includegraphics[width=0.4\textwidth]{../assets/greedy/stone_simple.png}
\end{center}

In der Grafik oben gibt es viele flache Flächen,
die aus unnötig vielen Dreiecken aufgebaut sind.
Wir könnten es wie folgt in Dreiecke einteilen,
um die Anzahl zu reduzieren:

\begin{center}
\includegraphics[width=0.4\textwidth]{../assets/greedy/stone_greedy.png}
\end{center}

Somit werden zum Beispiel nur 2 Dreiecke verwendet,
um eine 2x2 Fläche von Voxels zu bilden,
statt 8 Dreiecke.

Diese Optimierung wird \gqq{Greedy Meshing} genannt.

{ \subsection{Binäres Greedy Meshing}

Ich habe erst geplant eine einfachere Implementation
von Greedy Meshing zu machen, bevor ich es mit
einem binären Algorithmus mache, aber dadurch,
dass das Culling schon binär ist, ist die binäre
Greedy Meshing Implementation sogar einfacher als
eine andere.

Der Grundgedanke in dieser Implementation ist,
dass jedes Mal, wenn wir eine Seite betrachten,
wir versuchen diese erst in eine Richtung so weit
zu erweitern, bis keine Seite mehr da ist und dann
erweitern wir diesen ganzen Streifen in die andere
Richtung.
Während man das macht, entfernt man immer die
Bits in der Bitmaske, die gerade für diese Dreiecke
verwendet werden, damit sie nicht später für andere
Dreiecke wieder verwendet werden.

% TODO include some images as a visualization of this

Wenn wir eine Bitmaske haben, in der ein
32-Bit Integer eine Reihe von Seiten darstellt,
dann geht es sehr schnell die Streifen zu erkennen,
da man mit den x86-Befehlen
\href{https://www.felixcloutier.com/x86/bsf}{BSF} \cite{bsf}
und
\href{https://www.felixcloutier.com/x86/bsr}{BSR} \cite{bsr}
die Anzahl von Nullen oder Einsen am Anfang oder
Ende eines 32-Bit Integers mit einem einzigen
Befehl berechnen kann.
Zudem kann man mit einem einfachen bitweisen ODER
überprüfen, ob der Streifen in die andere Richtung
erweitert werden kann, und man kann die Einträge
in der Bitmaske mit einem bitweisen UND entfernen.

% TODO clean up this code
\begin{lstlisting}[language=Rust]
let mut mask_copy = array[j];
let mut k = 0;
while mask_copy != 0 {
	let zeros = mask_copy.trailing_zeros();
	mask_copy >>= zeros;
	k += zeros;

	let ones = mask_copy.trailing_ones();
	mask_copy = mask_copy.checked_shr(ones).unwrap_or(0);
	let from = k;
	k += ones;

	// this entire strip of blocks, as a bitmask.
	// `<<` doesnt overflow in the way you would expect,
	// so we use `checked_shl` here instead.
	// `from != 32`, because otherwise `mask_copy == 0`, so the
	// left shift there will never overflow (in any problematic way).
	let ones_exp2 = 1_u32.checked_shl(ones).unwrap_or(0);
	let strip_mask = (ones_exp2.wrapping_sub(1)) << from;

	// expand the strip into a rectangle,
	// and unset any bits along the way
	array[j] &= !strip_mask;
	let mut strip_expand = 0;
	for next_mask in &mut array[(j + 1)..] {
		if *next_mask & strip_mask == strip_mask {
			strip_expand += 1;
			*next_mask &= !strip_mask;
		} else {
			break;
		}
	}

	// ...create a face
}
\end{lstlisting}

Vielleicht ist ihnen aber schon aufgefallen,
dass die Bitmaske, die wir für das Binäre Culling
verwendet haben nicht in die richtige Richtung
ausgerichtet ist.
Dabei ist nämlich ein Integer eine Reihe von Voxels,
die sich gegenseitig verdecken können, während wir
hier eine Reihe von Voxels
(oder genauer: sichtbare Seiten)
brauchen, die nebeneinander sind.
Deswegen müssen wir aus der Culling Bitmaske
eine Bitmaske konstruieren,
die getauschte Koordinaten hat.

% FIXME might be a bit too weird to present this after
% code above, considering this gets exectuted first
\begin{lstlisting}[language=Rust]
let culled_blocks_mask = // maske von Binaeres Culling
let mut greedy_mask = Box::<FaceMap<ChunkArray2D<u32>>>::default();
for (face, array2d) in culled_blocks_mask.iter_face() {
	for (i, array) in array2d.iter().enumerate() {
		for (j, mask) in array.iter().enumerate() {
			let mut mask = *mask;
			let mut k = 0;
			while mask != 0 {
				let zeros = mask.trailing_zeros();
				mask = (mask >> zeros) & !1;
				k += zeros;
				greedy_mask[face][k as usize][i] |= 1 << j;
			}
		}
	}
}
greedy_mask
\end{lstlisting}
% FIXME ensure that this footnote appears on the
% same page as the code above
\footnote{
	Da horizontale Streifen von Voxels häufiger
	vorkommen als vertikale Streifen, mache ich,
	dass die Bitmasken in den $x$- und $z$-Achsen
	beide horizontal ausgerichtet sind.
	Deswegen wird in der wirklichen Implementation
	für die $z$-Achse \code{i} und \code{j} getauscht.
}

Wenn wir nun diese Bitmaske haben, können wir mit dem
oben genannten Algorithmus die Seiten zu größeren
Seiten kombinieren.
Wenn wir also diese Information benutzen,
um die Dreiecke größer zu machen sind wir
schon mit Greedy Meshing fertig!

\vspace{0.5cm}

% Greedy Meshing 1 stats from: bench 11
\benchgraph{4}{0.25}{
	algo                   & blue       & red       \\
	Erste Implementation   & 12.834184  & 0.672563  \\
	Binäres Culling 1      &  2.883930  & 5.155590  \\
	Binäres Culling 2      &  8.524388  & 0.448866  \\
	Greedy Meshing 1       &  5.641821  & 0.425993  \\
}

% TODO recheck if this count is correct.
% the improvement seems lower than expected.
Mit diesem Algorithmus erhalten wir sogar bessere
Performance, da wir weniger Dreiecke konstruieren
müssen.
Zudem ist die Anzahl von Dreiecken in einer
typischen Spielwelt jetzt von
1.857.984 Ecken und 928.992 Dreiecken
runter auf 809.484 Ecken und 404.742 Dreiecken
gesunken.
Somit haben wir etwa 2,3-mal weniger Dreiecke.

\vspace{0.7cm}

{
	\setlength{\parindent}{0pt}
	... eigentlich sind wir aber noch nicht ganz fertig!\\
	Wir haben noch ein Problem übersehen.
}
 }

\pagebreak

{ % TEMP
\pagebreak

\subsection{Korrektur der Texturen}

Vor der Greedy Meshing Implementation sah eine
typische Spielwelt so aus:

\begin{center}
\includegraphics[width=0.8\textwidth]{../assets/greedy/landscape_normal.png}
\end{center}

Aber jetzt mit der neuen Implementation
sieht die gleiche Spielwelt wie folgt aus:

\begin{center}
\includegraphics[width=0.8\textwidth]{../assets/greedy/landscape_stone_only.png}
\end{center}

Das Problem ist, dass wir noch nicht überlegt haben,
wie die Texturen der Voxels in das Polygonnetz des
Chunks eingebaut werden.
Bevor wir Greedy Meshing benutzt haben, konnte man
einfach jedem Dreieck eine Textur geben, basierend
darauf zu welchem Voxel es gehört.
Jedoch kann jetzt ein Dreieck für mehrere Voxels
zuständig sein.

In \cite{yt_bin_greedy_mesher} und
\cite{gh_bin_greedy_mesher} wurden hierfür separate
Polygonnetze und Bitmasken für jede Art von Voxel
verwendet (und es gab auch keine Texturen, sondern
nur Farben).
In meinem Spiel will ich jedoch viele unterschiedliche
Arten von Voxels haben. Deswegen wäre es unrealistisch
ein separates Polygonnetz für jede Art zu haben.
Meine Strategie um dieses Problem zu lösen ist es,
mehrere Texturen auf einem Dreieck anzuzeigen.

% if this is a bad place for a \pagebreak,
% then insert some \vspace here
\pagebreak

Um dies zu verstehen, müssen wir erst einmal angucken,
wie die Texturen in einem Polygonnetz ohne
Greedy Meshing angewandt werden.
Es wird erst ein Array von Texturen erstellt,
das alle möglichen Texturen von Voxels beinhaltet.
Beim Erstellen des Polygonnetzes bekommt
dann jedes Dreieck\footnote{
	In Wirklichkeit kann man mit WebGPU nicht
	Daten mit jedem Dreieck assoziieren,
	sondern nur mit jedem Vertex.
	Somit speichern wir diesen Index doppelt so
	oft, wie wir es brauchen.
	Aber es geht hier um nur sehr wenig Daten,
	also ist das nicht relevant.
}
in dem Polygonnetz einen Index,
den es benutzen kann, um die richtige Textur für
dieses Voxel in dem Array zu finden.

Die einfachste Lösung wäre also, anstatt nur
einen Index zu speichern, ein 2 dimensionales Array
von Indexen zu speichern für jeden Voxel,
der von diesem Dreieck dargestellt wird.
Leider geht dies nicht, da alle Informationen,
die man für einzelne Dreiecke (also nicht das
gesamte Polygonnetz) speichern will, müssen in dem
\href{https://gpuweb.github.io/gpuweb/#enumdef-gpuvertexformat}{\code{GPUVertexFormat}}\cite{gpu_vertex_format}
sein.
Dabei gibt es nur Datentypen mit konstanter Größe,
wir brauchen aber ein Array mit dynamischer Länge.
Zudem wäre es unrealistisch immer das größtmögliche
Array zu speichern und unbenutzte Werte zu haben,
wenn wir ein kleineres Dreieck haben, da das größte
Dreieck $32 \cdot 32 = 1024$ Voxels groß sein kann,
aber die meisten Dreiecke weniger als $4$ Voxels
groß sind.
Somit würden wir sehr viele Daten verschwenden.

Wir müssen also ein Array für das gesamte
Polygonnetz verwenden.
Mein erster Gedanke dabei war es ein \code{Vec}
rüber zu senden, um so ein Array mit dynamischer
Länge zu erhalten, jedoch meint Bevy:\\
\code{"runtime-sized array can't be used in uniform buffers"}\cite{no_runtime_sized}\\
... was komisch ist, da WebGPU eigentlich dynamische
Arrays unterstützen sollte.

Zum Glück gibt es aber einen anderen Weg Daten mit
einer dynamischen Länge zu haben:
Wir können eine Textur verwenden.
Um damit ein Array von Indexen (also \code{u32})
darzustellen, verwenden wir das Format\\
\href{https://docs.rs/bevy/0.15.0/bevy/render/render_resource/enum.TextureFormat.html#variant.R32Uint}{\code{TextureFormat::R32Uint}}\cite{r32uint}.
Dieses Format hat nur einen roten Kanal, der als
\code{u32} gespeichert wird.

Wenn die Textur 3 dimensional wäre könnte somit
jedes Dreieck einen $z$-Index speichern und die
$x$-$y$-Ebene von der Textur wäre die Textur des Voxels.
Jedoch muss bei einer 3 dimensionalen Textur jede
solche Ebene die gleiche Größe haben.
Deswegen müssen wir eine Dimension tiefer gehen.
Jedoch müssten wir bei einer 2 dimensionalen Textur
die genaue Anordnung der einzelnen Voxel Texturen
überlegen.
Also habe ich mich entschieden eine 1 dimensionale
Textur zu verwenden.
Dabei merkt sich jedes Dreieck bei welchem Index
in diesem Array es startet, und die Breite dieser
Seite, damit es weiß, wie weit es in dem Array nach
vorne springen muss, wenn es eine Textur, die weiter
oben ist, benutzen will.
 
Es gibt also erstens die 1 dimensionale Textur,
für diesen gesamten Chunk:

\begin{lstlisting}[language=WGSL]
@group(2) @binding(102) var face_texture_indices: texture_1d<u32>;
\end{lstlisting}

Und dann noch die zwei Werte, die jedes einzelne
Dreieck wissen muss:

\begin{lstlisting}[language=WGSL]
@location(7) start_texture_index: u32,
@location(8) rect_width: u32,
\end{lstlisting}

Diese werden dann im \code{fragment} Shader wie
folgt verwendet, um die wirkliche Textur an
dieser Position zu bekommen:

\begin{lstlisting}[language=WGSL]
// Berechne den Index, an der die wirkliche Textur
// dieses Voxels ist, basiert auf der Position (uv).
let uv_offset = u32(in.uv.x) + u32(in.uv.y) * rect_width;
let face_index = start_texture_index + uv_offset;
let block_index = textureLoad(face_texture_indices, face_index, 0);

// Bekomme die Farbe an dieser Position in
// der wirklichen Textur dieses Voxels.
pbr_input.material.base_color = textureSample(
	block_textures, block_texture_sampler, in.uv, u32(block_index.r)
);
\end{lstlisting}

Um \code{face\_texture\_indices} zu berechnen,
müssen wir beim Erstellen eines Polygonnetzes
alle Voxels in einer kombinierten Seite
zu einem Array hinzufügen.
Das verbraucht etwa 1,45 ms,
also erhalten wir folgende Performance für
diese Implementation:

\vspace{0.3cm}

% Greedy Meshing 2 stats from: bench 13
\benchgraph{5}{0.2}{
	algo                   & blue       & red       \\
	Erste Implementation   & 12.834184  & 0.672563  \\
	Binäres Culling 1      &  2.883930  & 5.155590  \\
	Binäres Culling 2      &  8.524388  & 0.448866  \\
	Greedy Meshing 1       &  5.641821  & 0.425993  \\
	Greedy Meshing 2       &  7.092519  & 0.317280  \\
}

Mit dieser Implementation können jetzt viele
unterschiedliche Texturen auf nur einem Dreieck
angezeigt werden:

\begin{center}
\includegraphics[width=0.5\textwidth]{../assets/greedy/many_textures.png}
\end{center}
 }
