\newcommand{\minipagespace}[1]{
	\begin{minipage}[c]{#1\textwidth}
		\ 
	\end{minipage}
}

\section{Culling}

Der erste Weg eine Voxel Engine zu optimieren wird
offensichtlich, wenn man einen Würfel aus $8$
Voxels betrachtet. Dabei beobachten wir, dass
die Hälfte der Seiten der Voxels nach innen schauen
und somit nicht sichtbar sind. Wenn man die
Seitenlänge dieses Würfels verdoppelt, multipliziert
man die Oberfläche mit $4$ und das Volumen mit $8$.
Somit entstehen immer mehr Seiten, die nach innen
schauen, umso größer das Objekt ist. Also wird
dies bei großen Objekten dazu führen, dass die
meisten Seiten nicht sichtbar sind.

\begin{center}
\begin{figure}[ht]
	\minipagespace{0.04}
	\begin{minipage}[c]{0.4\textwidth}
		\begin{center}
\includegraphics[width=1\textwidth]{../assets/culling/opaque_8_blocks.png}
		\end{center}
	\end{minipage}
	\minipagespace{0.09}
	\begin{minipage}[c]{0.4\textwidth}
		\begin{center}
\includegraphics[width=1\textwidth]{../assets/culling/transparent_8_blocks.png}
		\end{center}
	\end{minipage}\hfill
\end{figure}
\end{center}

Mit \gqq{Culling} beschreibt man die Optimierung,
diese Seiten zu entfernen.

{ \subsection{Erste Implementation}

Die jetzige Methode ein Polygonnetz der Voxels
zu erstellen besteht darin, für jeden Voxel
ein Würfelnetz zu erstellen.
Um dies also zu cullen, überprüft man
noch die Nachbarvoxels, um zu entscheiden,
welche Seiten sichtbar sind, und erstellt nur
die sichtbaren Seiten.

\begin{figure}[ht]
	\begin{minipage}[c]{0.49\textwidth}
		\begin{center}
\includegraphics[width=0.8\textwidth]{../assets/culling/chunk_borders.png}
		\end{center}
	\end{minipage}
	\begin{minipage}[c]{0.49\textwidth}
Da die Spielwelt unendlich groß ist, muss sie
in viele einzelne Polygonnetze aufgeteilt werden.
Somit ist die Welt in sogenannte \gqq{Chunks}
eingeteilt, die $32 \times 32 \times 32$ Voxels
beinhalten. Das Polygonnetz, das die Voxels in einem
Chunk anzeigt, werde ich als
\gqq{Chunknetz} bezeichnen.
	\end{minipage}\hfill
\end{figure}

Da das Erstellen eines Chunknetzes lang dauern kann,
soll das Spiel nicht darauf warten,
da dies sonst sichtbar beim Spielen wäre.
Deswegen werden die Chunknetze in separaten
\href{https://de.wikipedia.org/wiki/Thread_(Informatik)}{Threads}
\cite{wiki_thread} erstellt.
Zudem können somit mehrere Chunknetze gleichzeitig
erstellt werden.

Wir brauchen also erst Zeit, um den Thread zu starten,
und dann verwenden wir die meiste Zeit, um das
Chunknetz zu erstellen.
Wir erhalten folgende Performance:

\vspace{0.3cm}

% dont know how to get rid of this
% warning or what it means
% Erste Implementation stats from: bench 01
\benchgraph{1}{0.0}{
	algo                   & blue       & red       \\
	Erste Implementation   & 12.834184  & 0.672563  \\
}

\vspace{0.3cm}

Dadurch entsteht aber ein neues Problem: \\
Wenn mehrere Threads zugriff auf die gleichen Daten
haben, muss dieser Zugriff synchronisiert werden,
da sonst eine
\href{https://de.wikipedia.org/wiki/Wettlaufsituation}{Wettlaufsituation}
\cite{wiki_wettlauf} (genauer gesagt ein Data Race)
entstehen kann \cite{nomicon_races}.
Durch diese Synchronisierung müsste aber jeder Zugriff
zu Chunks auf andere Threads warten, was das gesamte
Spiel langsamer machen würde. Deswegen werden die
Daten der nötigen Chunks zu dem Thread rüberkopiert,
was langsam ist, da es sich hier über sehr große Daten
handelt.

Culling braucht Information aus den benachbarten
Chunks, um zu entscheiden, ob die Ränder des Chunks
sichtbar sind. Somit müssen die 6 Nachbarchunks
auch zu dem Thread rüberkopiert werden, was sehr viele
Daten sind. Man könnte zwar nur die Voxels
rüberkopieren, die am Rand des Chunks sind,
aber wir werden gleich eine Methode sehen,
die dieses Problem und andere auf einmal löst.

\vspace{0.3cm}

Ein weiteres Problem besteht darin, dass dieser
Algorithmus für jeden Voxel noch die 6 Nachbarn
betrachten muss. Somit wird jeder Voxel 7-mal
betrachtet.
 }

{ \subsection{Binäres Greedy Meshing}

Ich habe erst geplant eine einfachere Implementation
von Greedy Meshing zu machen, bevor ich es mit
einem binären Algorithmus mache, aber dadurch,
dass das Culling schon binär ist, ist die binäre
Greedy Meshing Implementation sogar einfacher als
eine andere.

Der Grundgedanke in dieser Implementation ist,
dass jedes Mal, wenn wir eine Seite betrachten,
wir versuchen diese erst in eine Richtung so weit
zu erweitern, bis keine Seite mehr da ist und dann
erweitern wir diesen ganzen Streifen in die andere
Richtung.
Während man das macht, entfernt man immer die
Bits in der Bitmaske, die gerade für diese Dreiecke
verwendet werden, damit sie nicht später für andere
Dreiecke wieder verwendet werden.

% TODO include some images as a visualization of this

Wenn wir eine Bitmaske haben, in der ein
32-Bit Integer eine Reihe von Seiten darstellt,
dann geht es sehr schnell die Streifen zu erkennen,
da man mit den x86-Befehlen
\href{https://www.felixcloutier.com/x86/bsf}{BSF} \cite{bsf}
und
\href{https://www.felixcloutier.com/x86/bsr}{BSR} \cite{bsr}
die Anzahl von Nullen oder Einsen am Anfang oder
Ende eines 32-Bit Integers mit einem einzigen
Befehl berechnen kann.
Zudem kann man mit einem einfachen bitweisen ODER
überprüfen, ob der Streifen in die andere Richtung
erweitert werden kann, und man kann die Einträge
in der Bitmaske mit einem bitweisen UND entfernen.

% TODO clean up this code
\begin{lstlisting}[language=Rust]
let mut mask_copy = array[j];
let mut k = 0;
while mask_copy != 0 {
	let zeros = mask_copy.trailing_zeros();
	mask_copy >>= zeros;
	k += zeros;

	let ones = mask_copy.trailing_ones();
	mask_copy = mask_copy.checked_shr(ones).unwrap_or(0);
	let from = k;
	k += ones;

	// this entire strip of blocks, as a bitmask.
	// `<<` doesnt overflow in the way you would expect,
	// so we use `checked_shl` here instead.
	// `from != 32`, because otherwise `mask_copy == 0`, so the
	// left shift there will never overflow (in any problematic way).
	let ones_exp2 = 1_u32.checked_shl(ones).unwrap_or(0);
	let strip_mask = (ones_exp2.wrapping_sub(1)) << from;

	// expand the strip into a rectangle,
	// and unset any bits along the way
	array[j] &= !strip_mask;
	let mut strip_expand = 0;
	for next_mask in &mut array[(j + 1)..] {
		if *next_mask & strip_mask == strip_mask {
			strip_expand += 1;
			*next_mask &= !strip_mask;
		} else {
			break;
		}
	}

	// ...create a face
}
\end{lstlisting}

Vielleicht ist ihnen aber schon aufgefallen,
dass die Bitmaske, die wir für das Binäre Culling
verwendet haben nicht in die richtige Richtung
ausgerichtet ist.
Dabei ist nämlich ein Integer eine Reihe von Voxels,
die sich gegenseitig verdecken können, während wir
hier eine Reihe von Voxels
(oder genauer: sichtbare Seiten)
brauchen, die nebeneinander sind.
Deswegen müssen wir aus der Culling Bitmaske
eine Bitmaske konstruieren,
die getauschte Koordinaten hat.

% FIXME might be a bit too weird to present this after
% code above, considering this gets exectuted first
\begin{lstlisting}[language=Rust]
let culled_blocks_mask = // maske von Binaeres Culling
let mut greedy_mask = Box::<FaceMap<ChunkArray2D<u32>>>::default();
for (face, array2d) in culled_blocks_mask.iter_face() {
	for (i, array) in array2d.iter().enumerate() {
		for (j, mask) in array.iter().enumerate() {
			let mut mask = *mask;
			let mut k = 0;
			while mask != 0 {
				let zeros = mask.trailing_zeros();
				mask = (mask >> zeros) & !1;
				k += zeros;
				greedy_mask[face][k as usize][i] |= 1 << j;
			}
		}
	}
}
greedy_mask
\end{lstlisting}
% FIXME ensure that this footnote appears on the
% same page as the code above
\footnote{
	Da horizontale Streifen von Voxels häufiger
	vorkommen als vertikale Streifen, mache ich,
	dass die Bitmasken in den $x$- und $z$-Achsen
	beide horizontal ausgerichtet sind.
	Deswegen wird in der wirklichen Implementation
	für die $z$-Achse \code{i} und \code{j} getauscht.
}

Wenn wir nun diese Bitmaske haben, können wir mit dem
oben genannten Algorithmus die Seiten zu größeren
Seiten kombinieren.
Wenn wir also diese Information benutzen,
um die Dreiecke größer zu machen sind wir
schon mit Greedy Meshing fertig!

\vspace{0.5cm}

% Greedy Meshing 1 stats from: bench 11
\benchgraph{4}{0.25}{
	algo                   & blue       & red       \\
	Erste Implementation   & 12.834184  & 0.672563  \\
	Binäres Culling 1      &  2.883930  & 5.155590  \\
	Binäres Culling 2      &  8.524388  & 0.448866  \\
	Greedy Meshing 1       &  5.641821  & 0.425993  \\
}

% TODO recheck if this count is correct.
% the improvement seems lower than expected.
Mit diesem Algorithmus erhalten wir sogar bessere
Performance, da wir weniger Dreiecke konstruieren
müssen.
Zudem ist die Anzahl von Dreiecken in einer
typischen Spielwelt jetzt von
1.857.984 Ecken und 928.992 Dreiecken
runter auf 809.484 Ecken und 404.742 Dreiecken
gesunken.
Somit haben wir etwa 2,3-mal weniger Dreiecke.

\vspace{0.7cm}

{
	\setlength{\parindent}{0pt}
	... eigentlich sind wir aber noch nicht ganz fertig!\\
	Wir haben noch ein Problem übersehen.
}
 }
