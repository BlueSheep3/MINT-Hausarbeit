\section{Einleitung}

Eine Voxel Engine ist zuständig Voxels auf den
Bildschirm zu projizieren. Voxels sind dabei wie
Pixel, nur 3D (der Name \gqq{Voxel} kommt von
\gqq{Volumen} kombiniert mit \gqq{Pixel}).
Also werden viele kleine Würfel zusammengesetzt,
um ein drei dimensionale Objekt darzustellen.

Dies wird in manchen Videospielen benutzt, wie
Minecraft. Als Spieleentwickler erstelle ich auch
ein Voxel Spiel und bin auf das Problem getreten,
wie man diese implementiert.
Es gibt hauptsächlich zwei Varianten,
wie Voxel Engines implementiert werden:

\begin{itemize}
	\item \href{https://de.wikipedia.org/wiki/Volumengrafik}{Volumengrafik}:
	Es wird ein komplett neues Verfahren entwickelt,
	um die Voxels zu rendern, welches direkt mit
	den Voxels arbeitet. Dies hat zwar gute
	Performance, hat aber auch die Limitation,
	dass alle Voxel nur einfache Würfel seien können.

	\item \href{https://de.wikipedia.org/wiki/Polygonnetz}{Polygonnetz}:
	Die Voxels werden zuerst in viele
	Dreiecke (oder allgemein Polygone) umgewandelt
	und dann von einem typischen 3D Renderer angezeigt.
	Somit muss man keinen neuen Renderer erstellen und
	kann auch andere Modelle als nur Würfel anzeigen,
	aber dieses Verfahren hat dafür schlechtere
	Performance.
\end{itemize}

Wegen der einfacheren Implementation, und dass man
komplexere Modelle als nur Würfel erstellen kann,
werden in den meisten Spielen die Polygonnetz
Variante bevorzugt.

\begin{figure}[ht]
	\begin{minipage}[c]{0.48\textwidth}
Mit der Polygonnetz Implementation sieht dann ein
Voxel wie rechts gezeigt aus.
Die grünen Linien zeigen dabei wie es in Dreiecke
eingeteilt ist.
	\end{minipage}
	\begin{minipage}[c]{0.5\textwidth}
		\begin{center}
\includegraphics[width=0.8\textwidth]{assets/SingleVoxel.png}
		\end{center}
	\end{minipage}\hfill
\end{figure}

\goodbreak

Ein Quadrat wird aus $2$ Dreiecken zusammengesetzt
und ein Würfel hat $6$ Seiten. Somit besteht ein Würfel
aus $12$ Dreiecken. Wenn der Spieler nur $100$ Voxels
weit sehen könnte, wäre der Durchmesser $200$ Voxels
und somit das gesamte Volumen das angezeigt werden
muss $200^3 = 8.000.000$ Voxels groß, also
$12 \cdot 200^3 = 96.000.000$ Dreiecke.

Wir sehen also, dass schon bei einer sehr kleinen
Sichtweite sehr viele Dreiecke erstellt werden.
Diese Hausarbeit beschäftigt sich deswegen mit der
Optimierung die Anzahl der Dreiecke so weit wie
möglich zu reduzieren.

Die meisten Optimierungen in dieser Hausarbeit
kommen von \url{https://youtu.be/qnGoGq7DWMc?si=Ke8vgcHWcgCdMGka}
und \url{https://github.com/TanTanDev/binary_greedy_mesher_demo},
und wurden dann noch ausgeweitet, damit sie auch für
Voxels mit Texturen funktionieren.
